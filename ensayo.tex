\documentclass[12pt]{article}

\begin{document}

\title{Optimizando la Clasificación de Texto Generado por Consumidores: Un Enfoque Multi-Etiqueta\thanks{Este ensayo se basa en la investigación publicada en "Research in Computing Science" del Instituto Politécnico Nacional.}}

\author{Jose Alfredo Monroy Sanchez\\
Ingeniero en Sistemas Computacionales\\
j.alfred.monroy@gmail.com}

\date{}

\maketitle

\begin{abstract}
  Este ensayo explora la investigación publicada en la revista científica "Research in Computing Science" del Instituto Politécnico Nacional (IPN), titulada "Learning Multi-Label Classification from Data Annotated with Unique Labels". El estudio aborda la crucial tarea de clasificar textos generados por consumidores, como comentarios, encuestas o correos de soporte, mediante un enfoque innovador que combina el aprendizaje de datos anotados con etiquetas únicas y la predicción de salidas multi-etiqueta junto con valores de confianza.
\end{abstract}

\section{Introducción}
Este ensayo examina la investigación reciente llevada a cabo por el Instituto Politécnico Nacional, que aborda la optimización de la clasificación de textos generados por consumidores.

\section{Hipótesis}
La investigación postula que la clasificación automática, mediante un enfoque de aprendizaje de datos anotados con etiquetas únicas y predicción de salidas multi-etiqueta, puede superar las limitaciones de la clasificación manual.

\section{Justificación}
La importancia de clasificar grandes volúmenes de texto generado por consumidores se ha incrementado, y esta investigación encuentra su justificación en la imposibilidad de realizar análisis manuales eficientes debido al volumen y velocidad de generación de estos textos.

\section{Problemática}
La problemática abordada se centra en las limitaciones de la clasificación manual y la discrepancia entre la necesidad de clasificación multi-etiqueta y la disponibilidad común de datos de entrenamiento.

\section{Conclusiones}
En conclusión, la investigación del Instituto Politécnico Nacional propone un enfoque innovador para la clasificación de textos generados por consumidores, ofreciendo nuevas perspectivas y avanzando significativamente en la comprensión de las dinámicas del consumidor en la era digital.

\end{document}
