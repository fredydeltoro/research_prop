\documentclass[12pt]{article}

\begin{document}

\title{\textbf{La identificación de subtipos de diabetes mellitus aplicando técnicas de análisis de conglomerados: una revisión sistemática\thanks{Este ensayo se basa en la investigación publicada en "International Journal of Environmental Research and Public Health".}}}
\author{Jose Alfredo Monroy Sanchez\\ Ingeniero en Sistemas Computacionales\\ j.alfred.monroy@gmail.com}
\date{\today}
\maketitle


\begin{abstract}
  Este documento es una revisión sistemática que analiza estudios que aplicaron técnicas de análisis de conglomerados para identificar subtipos de pacientes con diabetes. La revisión incluyó 14 estudios que analizaron variables como edad, índice de masa corporal, niveles de hemoglobina glicosilada, resistencia a la insulina, presencia de autoanticuerpos y otros factores clínicos en pacientes con diabetes tipo 1 y tipo 2.
\end{abstract}

\section{Introducción}
El tema principal de este articulo tiene como objetivo identificar nuevas subcategorías o subgrupos de pacientes con diabetes. El artículo aborda la complejidad de la diabetes, que va más allá de la clasificación tradicional en Tipo 1 y Tipo 2, y explora la posibilidad de identificar subgrupos más homogéneos mediante el uso de métodos de aprendizaje no supervisado, específicamente el análisis de conglomerados.

\section{Importancia}
Esta investigación tiene relevancia en la actualidad porque identifica subtipos de pacientes con diabetes a través de técnicas de análisis de conglomerados y relaciona estos subtipos con complicaciones específicas de la diabetes. Esto resalta la complejidad de la enfermedad y la necesidad de enfoques de tratamiento más personalizados. Por lo tanto, esta investigación puede tener implicaciones importantes en el desarrollo de tratamientos más efectivos y personalizados para pacientes con diabetes. 
Además, es importante tener en cuenta que esta revisión sistemática fue parte de la investigación financiada por la Universidad Nazarbayev (proyecto 080420FD1916), lo que demuestra el interés en el estudio y la importancia de la investigación en el campo.

\section{Objetivos}
Los objetivos observados se presentan acontinuación.

\textbf{Identificación de Estudios:} El objetivo principal es identificar los estudios de investigación que hayan aplicado técnicas de análisis de conglomerados (CA) no supervisado en pacientes con Diabetes Mellitus (DM).

\textbf{Revisión Sistemática:} Realizar una revisión sistemática de los estudios seleccionados que utilizan técnicas de análisis de conglomerados en pacientes con DM. La revisión se lleva a cabo siguiendo las pautas de PRISMA (Preferred Reporting Items for Systematic Reviews and Meta-Analyses).

\textbf{Búsqueda en Bases de Datos:} Utilizar estrategias de búsqueda específicas en bases de datos médicas, como Medline Complete y PubMed, para encontrar estudios relevantes en el área de análisis de conglomerados no supervisado de pacientes con DM.

\textbf{Selección de Estudios:} Realizar un proceso de selección, llevado a cabo por dos investigadores independientes, para identificar los estudios pertinentes. Este proceso incluye la eliminación de duplicados, la revisión de títulos y resúmenes, y la obtención y evaluación de los textos completos de los estudios potencialmente adecuados.

En resumen, la investigación tiene como objetivo explorar y resumir los estudios que han aplicado análisis de conglomerados no supervisados en pacientes con diabetes, proporcionando información detallada sobre los métodos utilizados, las variables consideradas y los resultados obtenidos en términos de identificación de subgrupos de pacientes diabéticos.

\section{Metodología}
La metodología adoptada en este estudio se caracteriza como cuantitativa, ya que se evidencia una revisión sistemática y una exhaustiva búsqueda en las bases de datos Medline Complete y PubMed mediante términos de búsqueda específicos. Este enfoque metodológico se fundamenta en la recopilación y análisis de datos cuantificables para obtener resultados objetivos. Se han establecido criterios rigurosos para la selección de estudios, destacando la inclusión de la población de interés, es decir, pacientes con Diabetes Mellitus (DM). Además, se emplearon algoritmos de agrupación no supervisada y se aprovecharon datos clínicos para llevar a cabo el proceso de agrupamiento, proporcionando así un marco metodológico robusto y orientado a obtener resultados cuantitativos significativos.

\section{Conclusiones Y Resultados}
Los principales hallazgos son los distintos subtipos de diabetes que se presentan a continuacion.

\textbf{Diabetes Autoinmune Grave:} (SAID, por sus siglas en inglés): caracterizado por un inicio temprano de la enfermedad, bajo índice de masa corporal (IMC), deficiencia de insulina y presencia de autoanticuerpos relacionados con la diabetes.

\textbf{Diabetes con Deficiencia Severa de Insulina:} (SIDD, por sus siglas en inglés): similar al subtipo SAID, pero sin la presencia de autoanticuerpos.

\textbf{Diabetes con Resistencia Severa a la Insulina:} (SIRD, por sus siglas en inglés): caracterizado por resistencia a la insulina y alto IMC.

\textbf{Diabetes Relacionada con la Obesidad Moderada:} (MOD, por sus siglas en inglés): asociada con obesidad y resistencia moderada a la insulina.

\textbf{Diabetes Relacionada con la Edad Moderada:} (MARD, por sus siglas en inglés): caracterizado por una edad avanzada y cambios metabólicos moderados.

En conclusión, este estudio sugiere que la segmentación más detallada de la diabetes en varios tipos puede desempeñar un papel crucial en el mejoramiento de los enfoques de tratamiento. Al identificar subgrupos más específicos, se facilita una delimitación más precisa de los síntomas y las complicaciones asociadas, permitiendo así estrategias de tratamiento más personalizadas y efectivas. Esta aproximación podría tener un impacto significativo en la calidad de vida de los pacientes, al proporcionar intervenciones más específicas y adaptadas a las características particulares de su tipo de diabetes.
\end{document}
